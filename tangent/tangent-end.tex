%接線に関する参考資料
%\documentclass{jsarticle} \usepackage{graphics} \usepackage{amsmath} \usepackage{amssymb} \usepackage{cases} \usepackage{ascmac}

%\newcommand{\pt}{$(x_1,y_1)$}


%\begin{document}
    \section{まとめ}
    これまでの話から、ある二次曲線の接線を求めるときその二次曲線上の点を\pt とすると以下の書き換えで接線の方程式が導出できるとわかった。

    \begin{shadebox}
        \begin{itemize}
            \item $x,y$が1次の項
            \[
            x,\hspace{3mm} y\rightarrow \frac{x+x_1}{2} ,  \hspace{3mm} \frac{y+y_1}{2}
            \]
            \item $x,y$が2次の項
            \[
            x^2, \hspace{3mm} y^2 \rightarrow x_1 x, \hspace{3mm} y_1 y
            \]
        \end{itemize}
    \end{shadebox}
    この置き換えを行う条件は二次曲線の方程式が完全に展開されていることであることを注意として述べておく。便利な計算の方法として覚えてもらいたい。



%\end{document}
