%接線に関する参考資料
%\documentclass{jsarticle} \usepackage{graphics} \usepackage{amsmath} \usepackage{amssymb} \usepackage{cases}

%\newcommand{\pt}{$(x_1,y_1)$}
%\newcommand{\ptt}{$(p,q)$}

%\begin{document}
    \section{円}
    \subsection{導出}
    円の接線の方程式を導出するにはまず円の中心が原点にある時のものを考える。そのあと中心が原点にない場合を考えるが、この時は図形の平行移動を利用して簡単に求める。
    円の中心が原点にあるときの接線の導出には大きく2つの方法が考えられる。1つ目は円の接線と接点への半径は垂直であることから接線の傾きを決定する方法。2つ目は陰関数の微分を使って傾きを決定する方法。傾きの決定の方法が違うのみである。

    では早速、中心が原点にある半径$r$の円
    \[
    x^2 +y^2=r^2
    \]
    について円上の点\pt における接線を求める。傾きの求め方は2種類あるで分けて説明する。
    \begin{description}
        \item[円の接線の性質から傾きを求める] この方法をとる場合はいくつか場合分けをしなければならない。
        \begin{enumerate}
            \item $x_1 \neq 0$かつ$y_1 \neq 0$の時\\
            この時、原点から点\pt までの傾きは$\frac{y_1}{x_1}$なので接線の傾きは$-\frac{x_1}{y_1}$
            \item $x_1=0$の時\\
            この時は接点への半径が$y$軸と平行になるわけだから接線の傾きは$0$($x$軸と平行)となる。この結果は先の$x_1 \neq 0$かつ$y_1 \neq 0$の時の結果と合わせていいことがわかる。
            \item $y_1 = 0$の時\\
            この時は接点への半径が$x$軸と平行になるわけだから接線の傾きは$y$軸と平行\footnote{傾きが$y$軸と平行というのは厳密には傾きが定義できないということである。そのため多くの図形と方程式の問題で傾きが定義できない場合を特別視して場合分けする。}ということになる。
        \end{enumerate}
        以上から接線の傾きは$y_1 \neq 0$の時$-\frac{x_1}{y_1}$、$y_1 = 0$の時は傾きが定義できない、すなわち$y$軸と平行だということが分かった。
        \item[陰関数の微分から傾きを求める] 数I\hspace{-.1em}I\hspace{-.1em}Iの範囲であるため文系諸君は読み飛ばして構わない。\\
        \[
        x^2 +y^2=r^2
        \]
        の両辺を$x$で微分する。この方法でも$y_1 = 0$の場合を分けて考える。以下$y_1 \neq 0$である。
        \begin{eqnarray*}
            2x + 2y \cdot \frac{{\rm d}y}{{\rm d}x} &=& 0\\
            \Leftrightarrow \frac{{\rm d}y}{{\rm d}x} &=& -\frac{x}{y}
        \end{eqnarray*}
        よって\pt における接線の傾きは$-\frac{x_1}{y_1}$である。また$y_1 = 0$の時は先の方法と同様で接線の傾きは定義できない。
    \end{description}
    以上から接線の傾きを決定できた。ここから接線の方程式を求める。点\pt を通り傾き$-\frac{x_1}{y_1}$の直線の方程式は
    \begin{equation}
        y-y_1 = -\frac{x_1}{y_1} (x-x_1)
        \label{eq:2-1}
    \end{equation}
    確かに式(\ref{eq:2-1})で十分ではあるがここからさらに式変形を加えたい。
    \begin{eqnarray*}
        y-y_1 &=& -\frac{x_1}{y_1} (x-x_1)\\
        \Leftrightarrow y_1 y -y_1^2 &=& x_1^2 -x_1 x\\
        \Leftrightarrow x_1 x +y_1 y &=& x_1^2 +y_1^2\\
        &=& r^2
    \end{eqnarray*}
    以上の式変形から接線の方程式が
    \begin{equation}
        x_1 x +y_1 y= r^2
        \label{eq:2-2}
    \end{equation}
    であるとわかった。$y_1 = 0$の時は点$(x_1,0)$を通り$y$軸と平行なのだから接線の方程式は
    \[
    x=x_1
    \]
    となる。そもそもこの時の半径は$r=x_1$となるわけだから
    \[
    x=r
    \]
    となり、これは式(\ref{eq:2-2})によってあらわされる直線の1つといえる。

    このまま中心\ptt 、半径$r$の円のある円上の点\pt における接線の方程式を求める。まず円の方程式は
    \[
    (x-p)^2 +(y-q)^2=r^2
    \]
    である。このままだと式(\ref{eq:2-2})はうまく使えないため平行移動を使って円の中心と接点の両方を無理やり都合のいいところに移動する。すると円の方程式は
    \[
    x^2+y^2=r^2
    \]
    接点は$(x_1-p,y_1-q)$になる。こうなると式(\ref{eq:2-2})より接線の方程式は
    \[
    (x_1-p)x +(y_1-q)y=r^2
    \]
    となる。しかしこれは平行移動した後の結果であるため円の中心を\ptt に戻すようにもう一度平行移動しなければならない。ようやく答えは、
    \begin{equation}
        (x_1-p)(x-p) +(y_1-q)(y-q)=r^2
        \label{eq:2-3}
    \end{equation}
    このようになった。


    \subsection{確認}
    さて、放物線の方程式導出で明らかになった書き換えは覚えているだろうか。円の中心が原点にあるときの接線に関しては式(\ref{eq:2-2})から規則が成立しているのは明らかだが中心が原点にない場合を確認する。
    \[
    (x-p)^2 +(y-q)^2=r^2
    \]
    これを展開すると
    \[
    x^2-2px+y^2-2qy+p^2+q^2=r^2
    \]
    となる。この形を覚えてもらいたい。そして式(\ref{eq:2-3})を式変形する。
    \begin{eqnarray*}
        (x_1-p)(x-p) +(y_1-q)(y-q) &=& r^2\\
        \Leftrightarrow x_1 x -p(x +x_1) +p^2 +y_1 y -q(y+y_1) +q^2 &=& r^2\\
        \Leftrightarrow x_1 x -2p \cdot \frac{x +x_1}{2} +y_1 y -2q \cdot \frac{y+y_1}{2} +p^2 +q^2 &=& r^2
    \end{eqnarray*}
    これで規則の成立を確認できた。


%\end{document}
